\section{Diary}
	\subsection{Dec 2016}
		In the session before the Christmas break I decided that I wanted to create a simulation of Gravity. I decided on this because it seemed like a do-able project which would encompass a number of physical concepts (Newtonian gravity, constant acceleration formulae, etc) and some interesting programming concepts.
		
	\subsection{9th Jan 2017}
		In this session I fleshed out my idea. I decided to use OpenGL for rendering the simulation. I also implemented the core physics of the simulation: the gravity. I did this by first creating the Particle class and then implementing forces and movement. I then had every Particle exert a gravitational force on every other Particle.
	
	\subsection{16th Jan}
		I started work on the rendering system so that I could visualise my simulation. I used OpenGL for this. In this session I only got as far as simply rendering the circles for the particles
		
	\subsection{23rd Jan}
		By this session I had completed the core rendering with particles, forces and paths all being rendered to the screen. I then decided to try and implement elastic collisions. This ended up being more difficult than I had thought so I worked on this for a while.
	
	\subsection{30th Jan}
		In this session I continued working on the collisions but I also implemented a command-line system so that the user could control the simulation. This included all the commands listed in the \textbf{User Control} section.
	
	\subsection{6th Feb}
		In this session I implemented some sort of working collisions. Whilst they are not as good as I would have liked they work. I also created in this session a way of outputting the path of a particle over time to a \code{.csv} file.