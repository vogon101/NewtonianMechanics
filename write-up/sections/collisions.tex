\section{Momentum  and Collisions}
	Having implemented gravity I decided to add in simple 2D collisions to make the simulation slightly more realistic. This turned out to be more difficult than I had expected as many of the equations used for this are very complicated when used with vectors in two dimensions.
	
	I found online\footnote{Source: \url{http://vobarian.com/collisions/2dcollisions2.pdf}} a set of equations designed for fully elastic vector collisions. These are derived from two equations: Conservation of momentum and of kinetic energy:
	
	\begin{figure}[h]
		\begin{equation}
		\frac{1}{2}m_{1}u_{1}^2 + \frac{1}{2}m_{2}u_{2}^2 = \frac{1}{2}m_{1}v_{1}^2 + \frac{1}{2}m_{2}v_{2}^2
		\end{equation}
		
		\begin{equation}
		m_{1}u_{1} + m_{2}u_{2} = m_{1}v_{1} + m_{2}v_{2}
		\end{equation}
		\caption{Conservation of momentum and kinetic energy equations}
		\label{fig:movEqn}
	\end{figure}

	These equations give two equations that allow me to work out the velocity of the particles after the collision. However, this only works in one-dimension so to find the collision between particles in my simulation I first have to transform the velocity vectors so that the collision is as if it were in 1D. 
	
	This is done by taking the unit vectors in the directions normal and tangent to the collision. It is then possible to find the velocities of each particle in these directions by taking the dot product of the vectors. In the following equations $\vec{normal}$ and $\vec{tangent}$ are the unit tangent and normal vectors respectively.
	
	I will only consider the first particle as because of Newton's Second Law of motion the force produced by this collision will be the same on both particles. This means that once I find the final velocity of the first particle I can find the change in momentum and thus the magnitude of the force by using $F = \frac{\Delta p}{\Delta t}$
	
	\begin{figure}[h!]
		\begin{equation}
			u_{n1} = \vec{normal} \cdot \vec{u_1}
		\end{equation}
		\begin{equation}
			v_{t1} = \vec{tangent} \cdot \vec{u_1}
		\end{equation}
		\begin{equation}
			u_{n2} = \vec{normal} \cdot \vec{u_2}
		\end{equation}
		\caption{Equations to find the normal and tangent components of the velocity}
		\label{fig:compEqn}
	\end{figure}
		
	As the velocity in the tangent direction remains the same after the collision, I only need to deal with the normal velocity. It is then possible to calculate the final normal component to the velocity with the following formula (derived from the conservation of momentum and kinetic energy).
	\begin{figure}[h!]
		\begin{equation}
			v_{n1} = \frac{u_{n1}(m_1 - m_2) + 2 m_2 u_{n2}}{m_1+m_2}
		\end{equation}
		\caption{Equation to find the final velocity of particle 1 in the normal direction}
		\label{fig:colEqn}
	\end{figure}

	\newpage
	
	Finally I transform the two one-dimensional vectors into a single 2D vector velocity.
	
	\begin{figure}[h]
		\begin{equation}
			\vec{v_{n1}} = \vec{normal} * v_{n1}
		\end{equation}
		\begin{eqnarray}
			\vec{v_{t1}} = \vec{tangent} * v_{t1}
		\end{eqnarray}
		\begin{equation}
			\vec{v_1} = \vec{v_{n1}} + \vec{v_{t1}}
		\end{equation}
		\caption{Combination of the normal and tangent components of the velocity into a final vector}
		\label{fig:movEqn}
	\end{figure}

	As my simulation is built around forces, I want to find the forces produced from this. To do this I use the equation above to calculate the magnitude and then create two forces as vectors of that length: One in the direction of $p_1-p_2$ (where $p_1$ is the position vector of particle 1) and one in the opposite direction.

	The code that calculates this is listed below:
	
	\begin{figure}[h]
		\centering
		\begin{lstlisting}[language=Scala]
/**
* Calculate the forces generated by the collision of this particle and another
* @param that The particle to collide with
* @return List of the forces generated
*/
def collide (that: Particle): List[Force] = {
	
	val normal = (that.position - this.position).normalize
	val tangent = normal.tangent
	
	val u1 = this.velocity
	val u2 = that.velocity
	
	val m1 = this.mass
	val m2 = that.mass
	
	val u1n: Double = u1 dot normal
	val u1t: Double = u1 dot tangent
	val u2n: Double = u2 dot normal
	
	val v1t = u1t
	
	val v1n = ( ((m1-m2) * u1n) + (2 * m2 * u2n) ) / (m1 + m2)
	
	val vect_v1n = normal * v1n
	val vect_v1t = tangent * v1t
	
	val v1 = vect_v1n + vect_v1t
	
	val impulse1 = ((v1 * m1) - (u1 * m1)).length / DELTA_TIME
	
	if (impulse1 < 0.0000000001) {
		List()
	} else {
	
		val forces = List(
		Force(this, impulse1, (this.position - that.position).theta, alwaysDraw = true),
		Force(that, impulse1, (that.position - this.position).theta, alwaysDraw = true)
		)
		
		forces
	}
}
		\end{lstlisting}
		\caption{The code that calculates the result of collisions}
		\label{fig:collCode}
	\end{figure}

