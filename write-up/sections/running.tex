\section{Running the simulation}
	To run the simulation yourself a runnable build is available at \url{https://static.vogonjeltz.com/physics/} for windows. To run it: download the zip file and extract it. Then double click on the \code{.bat} file. The simulation requires java to work.
	
	Once the simulation has started, click on the window to enable control. You can then use the key-bindings and commands listed in the \textbf{User Control} section. The default simulation is a pool-like set-up.
	
	To run another simulation use the \code{run} command. The simulations available are listed below:
	
	\begin{figure}[h]
		\centering
		\begin{tabular}{ | l | l |}
			\hline
			\textbf{Name} & \textbf{Description} \\ \hline
			test & A test of the collision code \\ \hline
			collision & A simple demonstration of a 1D collision \\ \hline
			pool & A pool-like set-up but with larger balls and higher masses \\ \hline
			pool2 (default) & The default pool simulation \\ \hline
			orbit & A three-particle orbit simulation \\ \hline
			earth & A simulation of earth's orbit that outputs to a \code{.csv}\\ \hline
		\end{tabular}
		\caption{The simulations available to run}
		\label{table:simulations}
	\end{figure}